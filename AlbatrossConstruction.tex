\documentclass{article}
\usepackage[margin=0.5in]{geometry}

\begin{document}
\title{Albatross Avionics Build Notes}
\author{Madison Harrington}
\maketitle



\section{Propellor} \\

The shared files provided helpful photographs of how the propellor assembly should look. Our kit included more pieces than the applied aeronautics' team had and had one piece entirely missing. Our construction process was the following: 

\begin{itemize}

\item Use a dremel to allow screw to fit through propellor. Careful to stay even, in a circular shape, and to only take off as much as needed. 
\item Layer all the pieces together as shown in the picture, the circular disk should be on bottom, the screw through that. Use the washers
on top of the large circular disk, then the propellor, and a nut on top of that. Ensure it gets tightened on really well. 
\item Choose the smallest screw (or the largest that fits to the bottom of the large screw). This will be placed through the nose cap 
along with a small nut and txtbf{into the large screw} to secure the nose cap to the rest of the pieces. Ensure this is tightened well.

\end{itemize}



\section{Servos}

\begin{itemize} 

\item Servo Holders

   \begin {itemize}
   
   \item The given AA documentation provided 3D print files for three different sized servo holders, the team will print these    
   pieces off and use them to attach the servos to the Albatross
   
   \end{itemize}
   
\end{itemize}  

\end{document}
