\documentclass{article}
\usepackage[margin=0.5in]{geometry}
\usepackage{graphicx}

\begin{document}
\title{Albatross Avionics Build Notes}
\author{Madison Harrington}
\maketitle

\section{Propellor and Motor}

The construction of the Propellor was based on the shared documenation by Applied Aeronautics. They supplied instructive 
photographs of how it should look. Our kit inclluded more pieces than the applied aeronautics team had, and had one piece entirely missing. 

\begin{enumerate}

\item Use a dremmel to allow screw to fit through propellor. Careful to stay even, in a circular shape, and to only take off as much as needed. 
\item Layer all the pieces together as shown in the picture, the circular disk should be on bottom, the screw through that. Use the washers
on top of the large circular disk, then the propellor, and a nut on top of that. Ensure it gets tighted on really well. 
\item Choose the smallest screw (or the largest that fits to the bottom of the large screw). This will be placed through the nose cap 
along with a small nut and \txtbf{into the large screw} to secure the nose cap to the rest of the pieces. Ensure this is tightened well.

